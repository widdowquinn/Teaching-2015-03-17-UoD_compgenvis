%% pairwise_genome_alignment.tex
%% Author: Leighton Pritchard
%% Copyright: James Hutton Institute
%% Pairwise genome alignment approaches

%
\begin{frame}
  \frametitle{Pairwise genome alignments}
  \textcolor{olive}{Genome sequence data gives us much more detail and power} \\
  Pairwise comparisons require alignment of similar regions.
  \begin{center}
    \includegraphics[width=\textwidth]{images/pairwise_genome_alignment}
  \end{center}  
\end{frame}

%
\begin{frame}
  \frametitle{Pairwise genome alignments}
  \textcolor{hutton_green}{Which genomes should you align (or not bother with)?} \\
  \textcolor{RawSienna}{For reasonable analysis, genomes should}:
  \begin{itemize}
    \item derive from a sufficiently \textcolor{red}{recent} common ancestor, so that \textcolor{hutton_purple}{homologous regions can be identified}
    \item derive from a sufficiently \textcolor{red}{distant} common ancestor, so that \textcolor{hutton_purple}{biologically meaningful changes are likely to be found}
  \end{itemize}
\end{frame}

%
\begin{frame}
  \frametitle{Alignment algorithms/programs}
  \textcolor{hutton_green}{I assume you're familiar with BLAST} \\
  (but, if not, see \url{supporting_information} subdirectory) \\~\\
  \textcolor{RawSienna}{Na\"{i}ve alignment algorithms are not appropriate}:
  \begin{itemize}
    \item Needleman-Wunsch: optimal global alignment
    \item Smith-Waterman: optimal local alignment
  \end{itemize}
  \textcolor{hutton_blue}{Cannot handle rearrangement} \\
  \textcolor{hutton_purple}{Computationally expensive}  
\end{frame}

%
\begin{frame}
  \frametitle{Alignment algorithms/programs}
  \textcolor{hutton_green}{Many whole-genome alignment algorithms proposed} \\
  Handle genome-scale evolutionary processes, scalable \\~\\
  \begin{itemize}
    \item \href{http://www.bx.psu.edu/~rsharris/lastz/}{LASTZ (http://www.bx.psu.edu/\~rsharris/lastz/)}
    \item \href{http://genome.ucsc.edu/goldenPath/help/blatSpec.html}{\textcolor{hutton_blue}{\textbf{BLAT} (http://genome.ucsc.edu/goldenPath)}}
    \item \href{http://mugsy.sourceforge.net/}{Mugsy (http://mugsy.sourceforge.net/)}
    \item \href{http://www.ncbi.nlm.nih.gov/blast/html/megablast.html}{\textcolor{red}{\textbf{megaBLAST} (http://www.ncbi.nlm.nih.gov/blast/)}}
    \item \href{http://mummer.sourceforge.net/}{\textcolor{red}{\textbf{MUMmer} (http://mummer.sourceforge.net/)}}
    \item \href{http://lagan.stanford.edu/lagan_web/index.shtml}{LAGAN (http://lagan.stanford.edu/lagan\_web/index.shtml)}
    \item WABA, etc?
  \end{itemize}
\end{frame}

%
\begin{frame}
  \frametitle{BLAT
  \footnote{\tiny{\href{http://dx.doi.org/10.1101/gr.229202
}{Kent (2002) \textit{Genome Res.} doi:10.1101/gr.229202
}}}
  }
  Broadly similar to BLAST \\~\\
  \textcolor{hutton_blue}{Main differences:}
  \begin{itemize}
    \item optimised to find \textcolor{hutton_purple}{only exact or near-exact matches} (speed)
    \item indexes the subject genome, and \textcolor{hutton_purple}{\textit{scans the query}}
    \item connects homologous match regions into a single alignment (BLAST reports separately)
    \item reports mRNA match intron-exon bounds exactly (BLAST tends to extend beyond bounds)
  \end{itemize}
  \textcolor{hutton_green}{\textbf{ADVANTAGES}: fast, exact exon bounds, UCSC integration}
  \textcolor{RawSienna}{\textbf{DISADVANTAGES}: less sensitive on remote/divergent sequences}
\end{frame}

%
\begin{frame}
  \frametitle{megaBLAST
  \footnote{\tiny{Zhang \textit{et al.} (2000) \textit{J. Comp. Biol.} \textbf{7}(1-2): 203-214
}}
  \footnote{\tiny{Korf \textit{et al.} (2003) \textit{BLAST} O'Reilly \& Associates, Sebastopol, CA
}}
  }
  Optimised for {\small(see \href{http://www.ncbi.nlm.nih.gov/blast/Why.shtml}{http://www.ncbi.nlm.nih.gov/blast/Why.shtml})}:
  \begin{itemize}
    \item \textcolor{hutton_green}{speed and genome-level searching}
    \item \textcolor{hutton_blue}{queries on large sequence sets}: "query-packing"
    \item \textcolor{hutton_purple}{long alignments of very similar sequences} (\url{dc-megablast} for divergent sequences)
  \end{itemize}
  Uses Zhang et al. greedy algorithm, \textbf{not BLAST algorithm} \\
  \textcolor{RawSienna}{BLASTN+ defaults to megaBLAST algorithm}
\end{frame}

%
\begin{frame}
  \frametitle{MUMmer
  \footnote{\tiny{\href{http://dx.doi.org/10.1186/gb-2004-5-2-r12
}{Kurtz \textit{et al.} (2004) \textit{Genome Biol.} doi:10.1186/gb-2004-5-2-r12
}}}
  }
  Conceptually completely different to BLAST/BLAT/megaBLAST \\
  \textcolor{RawSienna}{Uses \textit{suffix trees} for pattern matching}
  \begin{itemize}
    \item \textcolor{hutton_green}{Finds maximal exact matches}
    \item \textcolor{hutton_blue}{Memory use depends only on reference sequence size}
  \end{itemize}
  \begin{columns}[T] 
    \column{.5\textwidth} 
      \textcolor{hutton_purple}{Suffix Tree:}
      \begin{itemize}
        \item Constructed and searched in $O(n)$ time
        \item Useful algorithms are nontrivial
        \item \url{BANANAS$}
      \end{itemize}
    \column{.5\textwidth}
      \includegraphics[width=0.75\textwidth]{images/suffix_tree}
  \end{columns}    
\end{frame}

%
\begin{frame}
  \frametitle{MUMmer
  \footnote{\tiny{\href{http://dx.doi.org/10.1186/gb-2004-5-2-r12
}{Kurtz \textit{et al.} (2004) \textit{Genome Biol.} doi:10.1186/gb-2004-5-2-r12
}}}
  }
  \textcolor{RawSienna}{Process:}
  \begin{enumerate}
    \item \textcolor{hutton_green}{Identify non-overlapping set of maximal exact matches (MUMs: \textit{maximal unique matches})}
    \item \textcolor{hutton_blue}{Cluster MUMs into \textit{alignment anchors}}
    \item \textcolor{hutton_purple}{Extend between anchors to produce final (gapped) alignment}    
  \end{enumerate}
  \textcolor{RawSienna}{The approach is very flexible}
  \begin{enumerate}
    \item Used in a suite of programs (\url{mummer}, \url{nucmer}, \url{promer}, $\ldots$)
    \item Nucleotide and "conceptual protein" (sensitive!) alignments
    \item Used for comparisons, scaffolding, repeat detection
    \item Basis of other aligners and assemblers, (\url{Mugsy}, \url{AMOS}, $\ldots$)
  \end{enumerate}
  
\end{frame}