%% final_slides.tex
%% Author: Leighton Pritchard
%% Copyright: James Hutton Institute
%% Final slides

%
\begin{frame}
  \frametitle{What didn't I get to?}
  \begin{itemize}
    \item   \textcolor{hutton_green}{Genome-Wide Association Studies (GWAS)}
    \begin{itemize}
      \item   Try \href{http://genenetwork.org/}{http://genenetwork.org/} to play with some data
    \end{itemize}
    \item Prediction of regulatory elements, e.g.
    \begin{itemize}
      \item {\tiny\href{http://dx.doi.org/10.1038/nature01644}
                              {Kellis \textit{et al.} (2003) \textit{Nature} doi:10.1038/nature01644}}
      \item {\tiny\href{http://dx.doi.org/10.1101/gr.5592107}
                              {King \textit{et al.} (2007) \textit{Genome Res.} doi:10.1101/gr.5592107}}
      \item {\tiny\href{http://dx.doi.org/10.1186/1471-2105-9-455}
                              {Chaivorapol \textit{et al.} (2008) \textit{BMC Bioinf.} doi:10.1186/1471-2105-9-455}}
      \item {\tiny\href{http://genome.ucsf.edu/compmoby}
                              {CompMOBY http://genome.ucsf.edu/compmoby}}
    \end{itemize}
    \item   \textcolor{hutton_purple}{Detection of Horizontal/Lateral Gene Transfer (HGT/LGT), e.g.}
    \begin{itemize}
      \item {\tiny\href{http://dx.doi.org/10.1093/nar/gki187}
                              {Tsirigos \& Rigoutsos (2005) \textit{Nucl. Acids Res.} doi:10.1093/nar/gki187}}
    \end{itemize}
    \item   \textcolor{RawSienna}{Phylogenomics, e.g.}
    \begin{itemize}
      \item {\tiny\href{http://dx.doi.org/10.1038/nrg1603}
                              {Delsuc \textit{et al.} (2005) \textit{Nat. rev. Genet.} doi:10.1038/nrg1603}}
      \item {\tiny\href{https://phylogenomics.wordpress.com/software/amphora/}
                              {AMPHORA https://phylogenomics.wordpress.com/software/amphora/}}
    \end{itemize}
  \end{itemize}
\end{frame}

\begin{frame}
  \frametitle{Messages to take away}
  \begin{itemize}
    \item   \textcolor{hutton_green}{Comparative genomics is a powerful set of techniques for:}
    \begin{itemize}
      \item Understanding and identifying evolutionary processes and mechanisms
      \item Reconstructing detailed evolutionary history of a set of organisms
      \item Identifying and understanding common genomic features of organisms
      \item Providing hypotheses about gene function for experimental investigation
    \end{itemize}
  \end{itemize}
\end{frame}

\begin{frame}
  \frametitle{Messages to take away}
  \begin{itemize}
    \item \textcolor{hutton_blue}{A huge amount of data is available to work with}
    \begin{itemize}
      \item And it's only going to get much, much larger
    \end{itemize}
    \item \textcolor{hutton_purple}{Results feed into many areas of study:}
    \begin{itemize}
      \item Medicine and health
      \item Agriculture and food security
      \item Basic biology in all fields
      \item Systems and synthetic biology
    \end{itemize}
  \end{itemize}
\end{frame}

\begin{frame}
  \frametitle{Messages to take away}
  \begin{itemize}
    \item \textcolor{hutton_green}{Comparative genomics is comparisons}
    \begin{itemize}
      \item What is \textit{similar} between two genomes?
      \item What is \textit{different} between two genomes?
    \end{itemize}
    \item \textcolor{hutton_blue}{Comparative genomics \textit{is} evolutionary genomics}
    \begin{itemize}
      \item Lots of scope for improvement in tools
    \end{itemize}
    \item \textcolor{hutton_purple}{Tools that `do the same thing' can give different output}
    \begin{itemize}
      \item BLAST vs MUMmer
      \item RBBH vs MCL
      \item The choice of application matters for correctness and interpretation
    \end{itemize}
  \end{itemize}
\end{frame}

\begin{frame}
  \frametitle{Messages to take away}
  Comparative genomics is
  \begin{itemize}
    \item \textcolor{hutton_green}{Fun}
    \item \textcolor{hutton_blue}{Indoor work, in the warm and dry}
    \item \textcolor{hutton_purple}{Not a job that involves a lot of heavy lifting}
  \end{itemize}
\end{frame}