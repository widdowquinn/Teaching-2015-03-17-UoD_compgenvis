%% orthologue_prediction.tex
%% Author: Leighton Pritchard
%% Copyright: James Hutton Institute
%% A brief introduction to orthologues, and their prediction

% 
\begin{frame}
  \frametitle{Why focus on orthologues?
    \footnote{\tiny{Chen and Zhang (2012) \textit{PLoS Comp. Biol.} \href{http://dx.doi.org/10.1371/journal.pcbi.1002784}{doi:10.1371/journal.pcbi.1002784
    }}}  
    \footnote{\tiny{Dessimoz (2011) \textit{Brief. Bioinf.} \href{http://dx.doi.org/10.1093/bib/bbr057}{doi:10.1093/bib/bbr057
    }}}
    \footnote{\tiny{Altenhoff and Dessimoz (2009) \textit{PLoS Comp. Biol.} \textbf{5}:e1000262 \href{http://dx.doi.org/10.1371/journal.pcbi.1000262}{doi:10.1371/journal.pcbi.1000262
    }}}
  }
  Formalisation of the idea of \textit{corresponding genes} in different organisms. \\
  \textcolor{hutton_blue}{Orthologues serve two purposes:}
  \begin{itemize}
    \item \textcolor{hutton_green}{\textbf{Evolutionary equivalence}}
    \item \textcolor{hutton_purple}{\textbf{Functional equivalence}} (``The Ortholog Conjecture'')
  \end{itemize}
  Applications in comparative genomics, functional genomics and phylogenetics. \\
  \textcolor{RawSienna}{Over 30 databases attempt to describe orthologous relationships} (\href{http://questfororthologs.org/orthology_databases
}{http://questfororthologs.org/orthology\_databases})
\end{frame}

% Orthologue-finding methods
\begin{frame}
  \frametitle{Finding orthologues
    \footnote{\tiny{Kristensen \textit{et al}. (2011) \textit{Brief. Bioinf.} \textbf{12}:379-391 \href{http://dx.doi.org/10.1093/bib/bbr030}{doi:10.1093/bib/bbr030
    }}}
    \footnote{\tiny{Trachana \textit{et al}. (2011) \textit{Bioessays} \textbf{33}:769-780 \href{http://dx.doi.org/10.1002/bies.201100062}{doi:10.1002/bies.201100062
    }}}
    \footnote{\tiny{Salichos and Rokas (2011) \textit{PLoS One} \textbf{6}:e18755 \href{http://dx.doi.org/10.1371/journal.pone.0018755.g006}{doi:10.1371/journal.pone.0018755.g006
    }}}
  }
  Multiple methods and databases
  \begin{columns}[T]    \begin{column}{6cm}
      \begin{itemize}
        \item \textcolor{hutton_green}{\textbf{Pairwise genome}}
        \begin{itemize}
          \item \href{http://armchairbiology.blogspot.co.uk/2012/07/on-reciprocal-best-blast-hits.html}{RBBH} (aka BBH, RBH), \href{http://link.springer.com/protocol/10.1007/978-1-59745-515-2_7}{RSD}, \href{http://inparanoid.sbc.su.se/cgi-bin/index.cgi}{InParanoid}, \href{http://roundup.hms.harvard.edu/}{RoundUp}
        \end{itemize}
        \item \textcolor{hutton_blue}{\textbf{Multi-genome}}
        \begin{itemize}
          \item \textit{Graph-based}: \href{http://www.ncbi.nlm.nih.gov/COG/}{COG}, \href{http://eggnog.embl.de/}{eggNOG}, \href{http://cegg.unige.ch/orthodb7}{OrthoDB}, \href{http://orthomcl.org/orthomcl/}{OrthoMCL}, \href{http://omabrowser.org/cgi-bin/gateway.pl}{OMA}, \href{http://multiparanoid.sbc.su.se/}{MultiParanoid}
          \item \textit{Tree-based}: \href{http://www.treefam.org/}{TreeFam}, \href{http://www.ensembl.org/info/genome/compara/index.html}{Ensembl Compara}, \href{http://phylomedb.org/}{PhylomeDB}, \href{https://trac.nbic.nl/loft/}{LOFT}
        \end{itemize}
      \end{itemize}
    \end{column}
    \begin{column}{4cm}
      \includegraphics[height=0.575\textheight]{images/orthology_databases}      
    \end{column}
  \end{columns}
\end{frame}

