%% whole_genome_comparison_intro.tex
%% Author: Leighton Pritchard
%% Copyright: James Hutton Institute
%% Whole genome comparison introduction slides

%
\begin{frame}
  \frametitle{Whole genome comparisons}
  \Large{
    \textcolor{olive}{
      \textbf{
      Comparisons of one whole or draft genome with another \\
      (or many others)
      }
    }
  }
\end{frame}

%
\begin{frame}
  \frametitle{Whole genome comparisons}
  Minimum requirement: \textbf{two genomes} \\
  \begin{itemize}
    \item \textcolor{hutton_green}{Reference Genome}
    \item \textcolor{hutton_blue}{Comparator Genome}
  \end{itemize}
  The experiment produces a comparative result \textcolor{hutton_purple}{\textit{that is dependent on the choice of genomes}}.
\end{frame}

%
\begin{frame}
  \frametitle{Whole genome comparisons}
  Experimental methods mostly involve direct or indirect DNA hybridisation \\
  \begin{itemize}
    \item \textcolor{hutton_green}{DNA-DNA hybridisation (DDH)}
    \item \textcolor{hutton_blue}{Comparative Genomic Hybridisation (CGH)}
    \item \textcolor{hutton_purple}{Array Comparative Genomic Hybridisation (aCGH)}    
  \end{itemize}
\end{frame}

%
\begin{frame}
  \frametitle{Whole genome comparisons}
  Analogously, \textit{in silico} methods mostly involve sequence alignment \\
  \begin{itemize}
    \item \textcolor{hutton_green}{Average Nucleotide Identity (ANI)}
    \item \textcolor{hutton_blue}{Pairwise genome alignment}
    \item \textcolor{hutton_purple}{Multiple genome alignment}    
  \end{itemize}
\end{frame}